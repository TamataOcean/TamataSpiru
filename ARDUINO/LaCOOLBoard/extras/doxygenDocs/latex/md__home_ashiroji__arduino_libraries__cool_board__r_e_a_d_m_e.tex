\subsubsection*{What is this repository for?}

Quick summary \+:

\hyperlink{class_cool_board}{Cool\+Board} A\+PI is a set of libraries and wrappers to simplify the access and usage of all the capabilites of the \hyperlink{class_cool_board}{Cool\+Board}.

Version \+: 1.\+0

\subsubsection*{How do I get set up?}

Configuration \+:

1/\+Download and Install the Arduino I\+DE (\href{https://www.arduino.cc/en/Main/Software}{\tt https\+://www.\+arduino.\+cc/en/\+Main/\+Software} )

2/\+Download and Add the E\+S\+P8266 Hardware extension to Arduino (\href{https://github.com/esp8266/Arduino}{\tt https\+://github.\+com/esp8266/\+Arduino} )

/!\textbackslash{} If you already have Arduino I\+DE and the E\+S\+P8266 Hardware extension , make sure you upgrade them to the latest version.

3/\+Download and Add the \hyperlink{class_cool_board}{Cool\+Board} Library set to the Arduino I\+DE \+:

a)First Method \+:

-\/\+Open the Arduino I\+DE

-\/\+Go to the \char`\"{}\+Sketch\char`\"{} Menu

-\/\+Include Library $>$ Manage Libraries.

-\/\+Search for \hyperlink{class_cool_board}{Cool\+Board}

-\/\+Install

-\/\+Restart Arduino I\+DE

b)Second Method (if you have a Cool\+Board.\+zip file )\+:

-\/\+Open the Arduino I\+DE

-\/\+Go to the \char`\"{}\+Sketch\char`\"{} Menu

-\/\+Include Library $>$ \char`\"{}\+Add .\+Z\+I\+P Library\char`\"{}.

-\/\+Search for Cool\+Board.\+zip

-\/\+Click Open

-\/\+Restart Arduino I\+DE

c)Third Method (if you have the bitbucket/github link)\+:

-\/\+Open Arduino I\+DE $>$ File $>$ Preferences

-\/\+Check the \char`\"{}\+Sketch\+Book Location\char`\"{} path

-\/\+Go to the Arduino/libraries folder (if it doesn\textquotesingle{}t exist, create one )

-\/\+Clone the repo there ( git clone \char`\"{}bitbucket/github link\char`\"{} )

-\/\+Restart Arduino I\+DE

4/\+Download E\+S\+P8266\+FS Tool (\href{https://arduino-esp8266.readthedocs.io/en/latest/filesystem.html#uploading-files-to-file-system}{\tt https\+://arduino-\/esp8266.\+readthedocs.\+io/en/latest/filesystem.\+html\#uploading-\/files-\/to-\/file-\/system})

5/\+Optional but Heavily Recommended \+: Download the E\+S\+P8266 Exception Decoder ( download link \+:\href{https://github.com/me-no-dev/EspExceptionDecoder/releases/tag/1.0.6}{\tt https\+://github.\+com/me-\/no-\/dev/\+Esp\+Exception\+Decoder/releases/tag/1.\+0.\+6}) Install guide \+: \href{https://github.com/me-no-dev/EspExceptionDecoder}{\tt https\+://github.\+com/me-\/no-\/dev/\+Esp\+Exception\+Decoder}

Dependencies \+:

You need the following libraries to be able to use the \hyperlink{class_cool_board}{Cool\+Board} A\+PI\+:

-\/\+Arduino\+Json(\href{https://github.com/bblanchon/ArduinoJson}{\tt https\+://github.\+com/bblanchon/\+Arduino\+Json})

-\/\+Neo\+Pixel\+Bus(\href{https://github.com/Makuna/NeoPixelBus}{\tt https\+://github.\+com/\+Makuna/\+Neo\+Pixel\+Bus})

-\/\+Time\+Lib(\href{https://github.com/PaulStoffregen/Time}{\tt https\+://github.\+com/\+Paul\+Stoffregen/\+Time})

-\/\+D\+S1337\+R\+TC(\href{https://github.com/etrombly/DS1337RTC}{\tt https\+://github.\+com/etrombly/\+D\+S1337\+R\+TC})

-\/\+Dallas\+Temperature(\href{https://github.com/milesburton/Arduino-Temperature-Control-Library}{\tt https\+://github.\+com/milesburton/\+Arduino-\/\+Temperature-\/\+Control-\/\+Library})

Configuration Files \+:

-\/\+The \hyperlink{class_cool_board}{Cool\+Board} A\+PI heavily uses the S\+P\+I\+F\+FS for storing and retreiving configuration and data files

This is a description of the configuration files and what are they used for \+:

1/cool\+Board\+Config.\+json \+:

log\+Interval\+: The time Interval to wait,in seconds, between two logs

irene\+Active\+: Put this flag to 1(true) if you are using the \hyperlink{class_irene3000}{Irene3000} module

jetpack\+Active\+:Put this flag to 1(true) if you are using the \hyperlink{class_jetpack}{Jetpack} modue

external\+Sensors\+Active\+: Put this flag to 1(true) if you are using a supported external Sensor

sleep\+Active\+: Put this flag to 1(true) if you want your \hyperlink{class_cool_board}{Cool\+Board} to enable Sleep mode In Sleep Mode \+: your \hyperlink{class_cool_board}{Cool\+Board} will do the following \+: -\/read\+Sensors -\/activate Actors(if any) -\/log the data -\/check for updates -\/go to sleep for log\+Interval period of time

user\+Active\+:Put this flag to 1(true) if you want your \hyperlink{class_cool_board}{Cool\+Board} to collect user\+Data \+: user\+Name \hyperlink{class_cool_board}{Cool\+Board} M\+AC Address Time\+Stamp

manual /!\textbackslash{} \+: Put this flag to 1(true) to control to turn on/off your actors and bypass your initial actors configuration \begin{DoxyVerb}         Be Careful with this mode!!.

         When this mode is actif , the CoolBoard will not restart automatically to apply any new configuration sent via
         the update method.

         Restarting the coolBoard when in manual mode will deactivate all the actors!

         Never forget to reset this to 0(false) to go back to normal mode!
\end{DoxyVerb}


2/cool\+Board\+Led\+Config.\+json\+:

led\+Active\+: Put this flag to 1(true) if you want to have Light Effects from the on Board L\+ED

3/cool\+Board\+Sensors\+Config.\+json\+:

temperature \+: Put this flag to 1(true) if you want to collect the Temperature using the \hyperlink{class_b_m_e280}{B\+M\+E280} Sensor

humidity \+: Put this flag to 1(true) if you want to collect the humidity using the \hyperlink{class_b_m_e280}{B\+M\+E280} Sensor

pressure \+: Put this flag to 1(true) if you want to collect the pressure using the \hyperlink{class_b_m_e280}{B\+M\+E280} Sensor

visible \+: Put this flag to 1(true) if you want to collect the visible light index using the S\+I114X Sensor

ir \+: Put this flag to 1(true) if you want to collect the infrared light index using the S\+I114X Sensor

uv \+: Put this flag to 1(true) if you want to collect the ultraviolet light index using the S\+I114X Sensor

vbat \+: Put this flag to 1(true) if you want to collect the battery voltage

soil\+Moisture \+: Put this flag to 1(true) if you want to collect the soil Moisture

4/external\+Sensors\+Config.\+json\+:

sensors\+Number\+: the number of supported external sensors you connect to the cool\+Board

reference\+: the reference of a supported external sensor(e.\+g \hyperlink{class_n_d_i_r___i2_c}{N\+D\+I\+R\+\_\+\+I2C} , Dallas\+Temperature )

type\+: the type of the measurments you are making (e.\+g \+: co2, temperature,voltage ... )

address \+: the sensor\textquotesingle{}s address , if it has one (e.\+g \+: \hyperlink{class_n_d_i_r___i2_c}{N\+D\+I\+R\+\_\+\+I2C} C\+O2 sensor\textquotesingle{}s address is 77 )

5/irene3000\+Config.\+json\+:

water\+Temp.\+active\+: Put this flag to 1(true) in order to use the temperature sensor connected to the \hyperlink{class_irene3000}{Irene3000}

ph\+Probe.\+active\+: Put this flag to 1(true) in order to use the ph sensor connected to the \hyperlink{class_irene3000}{Irene3000}

adc2.\+active\+: Put this flag to 1(true) in order to use the extra A\+DC input of the \hyperlink{class_irene3000}{Irene3000}

adc2.\+gain\+: this is the value of the gain applied to the extra A\+DC input of the \hyperlink{class_irene3000}{Irene3000}

adc2.\+type\+: the type of the measurments you are making (e.\+g \+: co2, temperature,voltage ... )

6/jet\+Pack\+Config.\+json and cool\+Board\+Actor\+Config.\+json\+:

Act\mbox{[}i\mbox{]}.actif\+: Put this flag to 1(true) in order to use the jetpack output N�i (0..7)

Act\mbox{[}i\mbox{]}.inverted\+:Put this flag to 1(true) if the actor is inverted (e.\+g \+: a cooler is activated when Temp$>$Temp\+Max) Put this flag to 0(false) if the actor is not\+Inverted(e.\+g \+: a heater is activated when Temp$<$Temp\+Min)

Act\mbox{[}i\mbox{]}.temporal\+:Put this flag to 1(true) if you want the actor to be actif of a period of time , then inactif for another period of time.

Act\mbox{[}i\mbox{]}.type\+:\mbox{[}\char`\"{}primary\+Type\char`\"{},\char`\"{}secondary\+Type\char`\"{}\mbox{]} \+: this array contains the priamry\+Type and the secondary\+Type of the actor -\/\+The primary\+Type is the type associated to the sensors. (e.\+g \+: primary\+Type \+: \char`\"{}\+Temperature\char`\"{} is associated to the sensor of type \char`\"{}\+Temperature\char`\"{}). -\/\+It can also be empty , in case of P\+U\+RE temporal actors

\begin{DoxyVerb}                                             -The secondaryType is only used in temporal mode.
                                             it can be : -"" (empty):the actor will be on for a period of timeHigh ms
                                                                     the actor will be off for a period of timeLow ms

                                                         -"hour" :the actor will be on when the Hour is equal or greater then hourHigh
                                                                  the actor will be off when the Hour is equal or greater then hourLow

                                                         -"minute":the actor will be on when the Minute is equal or greater then minuteHigh
                                                                  :the actor will be off when the Minute is equal or greater then minuteLow

                                                         -"hourMinute":the actor will be on when : Hour == hourHigh AND Minute >= minuteHigh
                                                                                                   Hour > hourHigh
                                                                       the actor will be off when : Hour == hourLow AND Minute >= minuteLow
                                                                                                    Hour>hourLow

                                              /!\ NOTE  that if both primaryType and secondaryType are valid : the actor will be in
                                                 mixed mode : it will need to valid both Time and measurment conditions to go on or off

Act[i].low:[rangeLow,timeLow,hourLow,minuteLow] : this array contains the values previously described:
                                                  -rangeLow is the minimum of the range at which 
                                                  to activate(deactivate in inverted mode) the actor

                                                  -timeLow is the time spent off in temporal mode

                                                  -hourLow is the hour to turn off the actor when secondaryType is hour or hourMinute

                                                  -minuteLow is the minute to turn off the actor when secondaryType is minute or hourMinute


Act[i].high:[rangeHigh,timeHigh,hourHigh,minuteHigh]: this array contains the values previously described:
                                                  -rangeHigh is the maximum of the range at which 
                                                  to deactivate(activate in inverted mode) the actor

                                                  -timeHigh is the time spent on in temporal mode

                                                  -hourHigh is the hour to turn on the actor when secondaryType is hour or hourMinute

                                                  -minuteHigh is the minute to turn on the actor when secondaryType is minute or hourMinute


   Note that the coolBoardActorConfig.json contains only one Actor.
\end{DoxyVerb}


7/mqtt\+Config.\+json\+:

mqtt\+Server\+: This is the mqtt\+Server (ip/url) address

user\+: This is the user\+Id

buffer\+Size\+: This is the memory allocated to the mqtt buffer in bytes

in\+Topic \+: this is the topic that the cool\+Board subscribes to (receives updates from )

out\+Topic \+: this is the topic that the cool\+Board will publish to.

8/rtc\+Config.\+json\+:

time\+Server\+: N\+TP server ip address

local\+Port\+: port used to make the N\+TP request to update the time

9/wifi\+Config.\+json\+:

wifi\+Count\+: the number of wifis saved in this configuration file

time\+Out\+:access point timeout in seconds.

nomad\+: put this flag to 1(true) to activate nomad mode. in nomad mode \+: the cool\+Board will only try to connect to known Wi\+Fis. if it fails it will N\+OT lunch the access point.

How to run the exemples \+: \begin{DoxyVerb}-Open Arduino IDE 

-File > Exemples > CoolBoard

-Select the Exemple you want

-Flash it

-Flash the SPIFFS ( this is only required for the CoolBoardExemple, CoolBoardFarmExemple and CoolBoardStationExemple)

-Open The Serial Monitor

-Sit back and Enjoy!
\end{DoxyVerb}


\subsubsection*{Contribution guidelines}

For minor fixes of code and documentation, please go ahead and submit a pull request.

Larger changes (rewriting parts of existing code from scratch, adding new functions to the core, adding new libraries) should generally be discussed by opening an issue first.

Feature branches with lots of small commits (especially titled \char`\"{}oops\char`\"{}, \char`\"{}fix typo\char`\"{}, \char`\"{}forgot to add file\char`\"{}, etc.) should be squashed before opening a pull request. At the same time, please refrain from putting multiple unrelated changes into a single pull request.

\subsubsection*{License and credits}

All files under src/internal are modified versions of existing libraries. All Credit of the original work goes to their respective authors.

All Other files are provided as is under the M\+IT license \+:

Copyright (c) 2017 La Cool Co S\+AS

Permission is hereby granted, free of charge, to any person obtaining a copy of this software and associated documentation files (the \char`\"{}\+Software\char`\"{}), to deal in the Software without restriction, including without limitation the rights to use, copy, modify, merge, publish, distribute, sublicense, and/or sell copies of the Software, and to permit persons to whom the Software is furnished to do so, subject to the following conditions\+:

The above copyright notice and this permission notice shall be included in all copies or substantial portions of the Software.

T\+HE S\+O\+F\+T\+W\+A\+RE IS P\+R\+O\+V\+I\+D\+ED \char`\"{}\+A\+S I\+S\char`\"{}, W\+I\+T\+H\+O\+UT W\+A\+R\+R\+A\+N\+TY OF A\+NY K\+I\+ND, E\+X\+P\+R\+E\+SS OR I\+M\+P\+L\+I\+ED, I\+N\+C\+L\+U\+D\+I\+NG B\+UT N\+OT L\+I\+M\+I\+T\+ED TO T\+HE W\+A\+R\+R\+A\+N\+T\+I\+ES OF M\+E\+R\+C\+H\+A\+N\+T\+A\+B\+I\+L\+I\+TY, F\+I\+T\+N\+E\+SS F\+OR A P\+A\+R\+T\+I\+C\+U\+L\+AR P\+U\+R\+P\+O\+SE A\+ND N\+O\+N\+I\+N\+F\+R\+I\+N\+G\+E\+M\+E\+NT. IN NO E\+V\+E\+NT S\+H\+A\+LL T\+HE A\+U\+T\+H\+O\+RS OR C\+O\+P\+Y\+R\+I\+G\+HT H\+O\+L\+D\+E\+RS BE L\+I\+A\+B\+LE F\+OR A\+NY C\+L\+A\+IM, D\+A\+M\+A\+G\+ES OR O\+T\+H\+ER L\+I\+A\+B\+I\+L\+I\+TY, W\+H\+E\+T\+H\+ER IN AN A\+C\+T\+I\+ON OF C\+O\+N\+T\+R\+A\+CT, T\+O\+RT OR O\+T\+H\+E\+R\+W\+I\+SE, A\+R\+I\+S\+I\+NG F\+R\+OM, O\+UT OF OR IN C\+O\+N\+N\+E\+C\+T\+I\+ON W\+I\+TH T\+HE S\+O\+F\+T\+W\+A\+RE OR T\+HE U\+SE OR O\+T\+H\+ER D\+E\+A\+L\+I\+N\+GS IN T\+HE S\+O\+F\+T\+W\+A\+RE.

We can only gurantee that we did our best to have everything working on our side.

\subsubsection*{Who do I talk to?}

If you encounter a problem , have a good idea or just want to talk

Please open an issue, a pull request or send us an email \+:

La Cool Co \href{mailto:team@lacool.co}{\tt team@lacool.\+co} 